\chapter{Introduction}

In today's world, we are surrounded by complex systems. We use automated devices for many tasks in our daily life. In many cases, failure to ensure the correct functioning of these devices can lead to heavy losses. For example, an ATM machine working incorrectly can cause heavy financial loss. Similarly, incorrect functioning of the flight control systems installed in an aeroplane can put a risk on the lives of people aboard. Therefore, it is very important to ensure that a system functions correctly and has the desired properties. While system testing prior to usage can help find and resolve errors, it cannot ensure that the system is error free. To ensure that the system is free of design errors, formal specification and verification techniques come in play.

Formal specification and verification techniques work by mathematically modelling the system and its specification and then mathematically verifying that the system model has the desired properties. There are many approaches to formal specification and verification. One such approach is to use game theoretic concepts. Systems that involve multiple agents interacting with each other can be modelled as multi-player games. Game theoretic solution concepts can then be used to study the properties of such systems.

\section{Games, Strategies and Equilibria}

In this section, we briefly describe the terminology used in game theory literature and required for this thesis. We formally define these terms in Chapter 2.

\textit{\textbf{Games:}} A game consists of a finite number of players. Players play from among a number of available strategies. A combination of strategies, one for each player, decides the outcome of the game. Every outcome of the game has an associated payoff vector. A payoff vector is actually a tuple of real valued payoffs, one for each player. Players are rational and play to maximize their respective payoffs. Players may also be allowed to play mixed strategies (where they assign probabilities to each of the available strategies and then randomly select a particular strategy). In such cases expected values of payoffs are used to characterize the game. 

A game can be represented using a matrix which lists the payoff vectors for every possible combination of pure strategies (non-randomized strategies) of players. Such a representation is called normal form representation of a game. For example, Table \ref{table:rps} shows the normal form representation of a two player rock-paper-scissors game. The strategies of first player (or row player) are listed in rows and the strategies of second player (or column player) are listed in columns. The resulting payoff vectors are written in corresponding cells. In a payoff vector, the first value is the payoff received by row player and the second value is the payoff received by column player.

\begin{table}[h]
\centering
\caption{Rock-Paper-Scissors game in normal form.}
\label{table:rps}
\begin{tabular}{c c c c}
\noalign{\smallskip}\noalign{\smallskip}\noalign{\smallskip}\noalign{\smallskip}\hline\noalign{\smallskip}
         & Rock   & Paper  & Scissors \\ \noalign{\smallskip}\hline\noalign{\smallskip}
Rock     & 0 , 0  & -1 , 1 & 1 , -1   \\
Paper    & 1 , -1 & 0 , 0  & -1 , 1   \\
Scissors & -1 , 1 & 1 , -1 & 0 , 0    \\ \noalign{\smallskip}\hline
\end{tabular}
\end{table}

\textit{\textbf{Games played on graphs:}} Normal form representation of games is pretty exhaustive and is often inconvenient. If the number of players and available strategies is large, normal form representation is practically not possible. Various other succinct representations have been proposed. Games played on graphs is one such representation. Games played on graphs are very important and useful in formal modelling of systems and in formal specification and verification \cite{9}.

In this representation, a system is modelled as a state transition diagram. The states represent various configurations of the system. These games may be turn-based games or concurrent games. In turn-based games, in each state, only one player decides the next state. In concurrent games, in each state, every player chooses an action from a set of allowed actions. The combination of actions (one for each player) then determines the next state. The payoffs of the players are defined with respect to the runs (paths) in this graph. Players may have qualitative objectives (0-1 objectives) like reachability objectives, safety objectives, B{\"u}chi objectives etc. Players may also have quantitative objectives where a real valued payoff is associated to each run. This may be done by defining for each player, a preference ordering on the runs of the graph.

A strategy for a player is a mapping that associates every finite path in the graph to an action available to the player in the last state of that path. A strategy profile is a tuple of strategies, one for each player. A strategy profile determines the complete outcome of the game.

\textit{\textbf{Nash equilibrium:}} A Nash equilibrium is a strategy profile such that no player has a benefit on unilaterally deviating from its strategy. If mixed strategies are allowed, every game has at least one Nash equilibrium \cite{12}. In the rock-paper-scissors game shown in Table \ref{table:rps}, there is a mixed strategy Nash equilibrium when both the players randomize uniformly between the three pure strategies available to them.

\textit{\textbf{K-resilient Nash equilibrium:}} A \textit{k-resilient} Nash equilibrium is a strategy profile such that, if at most \textit{k} players deviate from their respective strategies, none of the deviating player can benefit. Such a strategy profile is \textit{resilient} to deviations by upto at most \textit{k} players \cite{Abraham-2006,Abraham-2008}. Thus, a Nash equilibrium is actually a \textit{1-resilient} Nash equilibrium. A \textit{k-resilient} Nash equilibrium may not exist in general \cite{Abraham-2006,Abraham-2008}.

\section{Our Contribution and Motivation}

\textit{\textbf{Contribution of this thesis:}} We consider non-deterministic multi-player concurrent reachability games as defined in \cite{BBM-concur10,BBM-report} and we prove some properties that characterize \textit{k-resilient} Nash equilibria in these games when only pure strategies are allowed. We then use these properties to develop algorithms for checking existence of \textit{k-resilient} Nash equilibrium in finite non-deterministic multi-player concurrent reachability games in untimed and timed settings when only pure strategies are allowed. For this purpose, we generalize the concept of \textit{suspect players} \cite{BBM-concur10,BBM-report,BBMU-fsttcs11,Romain-phd} and \textit{repellor sets} \cite{BBM-concur10,BBM-report,BBMU-fsttcs11} to \textit{k-suspect coalitions} and \textit{k-repellor sets} respectively. The properties that we prove also contain all the necessary information to compute a \textit{k-resilient} Nash equilibrium if it exists. Our algorithms, constructions and proofs are a natural generalization of those in \cite{BBM-concur10,BBM-report} used to characterize Nash equilibria in non-deterministic multi-player concurrent reachability games.

\textit{\textbf{Motivation:}} Although, a Nash equilibrium can tolerate unilateral deviations and hence be considered a stable strategy profile (considering that the players are rational), it is susceptible to deviations by more than one player. Players can form coalitions and deviate in order to increase their payoffs. This may disturb the stability of the system. Therefore, it makes sense to study solution concepts that are resistant to deviations by coalitions of players also. Although, one might assume that a coalition will deviate only if all the members of the coalition have some incentive (as the players are rational) and some equilibrium concepts have been proposed considering only such deviations \cite{Aumann-59,Bernheim-1987,Moreno-1996}, a \textit{k-resilient} Nash equilibrium is an even stronger notion as it also considers deviations where even only one player of the coalition can benefit. \citet{Abraham-2006} discuss several reasons to consider such deviations. There may be situations where only one player effectively controls the coalition. This can happen in a network if a player can hijack nodes in the network. A player may threaten other players or persuade them to deviate by promising side payments. This notion can also take care of situations where players make arbitrary moves. \citet{Abraham-2006,Abraham-2008} consider that there would be some kind of communication among the deviating coalition (as they use mediators to implement resilient strategies).

We claim that deviations by multiple players are possible even without any prior communication. We describe this scenario considering the risk taking behaviour of the players. Consider a strategy profile which is a Nash equilibrium. No player has an incentive on unilaterally deviating from its respective strategy. So, if communication between players is not allowed, it seems that this would be a perfectly stable strategy profile (as the players are rational). However, in real world situations, players might exhibit risk taking behaviour. If a player is not happy with his payoff, he might want to take a risk by deviating in the hope that some other player(s) would also deviate (reasoning similarly), and hence he may be benefited. While taking such risks, he might take into account his loss if any, in case no one else deviates and his deviation is unilateral. He may be ready to bear this loss considering his dissatisfaction with his current payoff and the benefits he may gain if the risk turns out in his favour. If the Nash equilibrium that we considered is a weak Nash equilibrium, a player may take such risks without the fear of any loss in case no one else deviates and his deviation is unilateral. For example, in games with qualitative objectives such as reachability games, a player can have the payoff of either 0 or 1. In such games, if a player has a payoff of 0 in the Nash equilibrium selected, he may like to deviate and take a risk because he has nothing to lose. Thus, the risk taking ability can be a threat to the stability of the system. Now consider a strategy profile which is a \textit{k-resilient} Nash equilibrium. In this case, if a player takes such a risk, it would be required that more than \textit{k} players deviate for the risk to turn out in his favour. Higher the value of \textit{k}, lower are the chances that the risk turns out in his favour. Therefore higher the resilience, lower is the risk taking ability of the players and hence more stable is the system. 

\section{Related Work}

Games played on graphs have been extensively used in formal specification and verification techniques. \citet{9} reviews some important work in this area. There has also been a lot of work on Nash equilibria and other related solution concepts (like subgame-perfect equilibria and secure equilibria) in turn-based multi-player games. \citet{Ummels-2008} present a very nice survey of the work in this area. An important result is that every deterministic turn-based game with Borel winning conditions has a pure strategy Nash equilibrium \cite{6}.
 
Recently, there has been a focus on multi-player concurrent games. The existence problem of Nash equilibrium in finite non-deterministic multi-player concurrent reachability games is NP-complete \cite{BBM-concur10,BBM-report,Romain-phd} whereas it is decidable in polynomial time for games with B{\"u}chi objectives \cite{BBMU-fsttcs11,Romain-phd}. For timed games with reachability objectives, existence problem of Nash equilibrium is EXPTIME-complete \cite{BBM-concur10,BBM-report,Romain-phd}. Similar results exist for games with safety objectives, co-B{\"u}chi objectives, Rabin objectives and parity objectives \cite{Romain-phd}.
 
Various solution concepts have been proposed to take care of deviations by more than one player. \citet{Halpern-2008,Halpern-2011} gives a brief review of these solution concepts. In \cite{Aumann-59}, a concept of strong equilibrium is proposed. A strong equilibrium is a strategy profile such that no coalition of players can deviate in such a way that all its members benefit. In \cite{Bernheim-1987}, coalition-proof Nash equilibrium is proposed which is a strategy profile such that no coalition has a self enforcing deviation. A self enforcing deviation is a deviation by a coalition such that all its members benefit from the deviation and no subset of this coalition has a further self enforcing deviation possible. The argument for considering only self enforcing deviations is that among the deviations considered for strong equilibrium, only self enforcing deviations may actually be taken. One may argue that if a deviation by a coalition is not self enforcing, players in the coalition would not take the deviation (as a subset of the coalition could cheat on them). According to the definitions, every strong equilibrium is also a coalition-proof Nash equilibrium. In \cite{Moreno-1996}, coalition-proof correlated equilibrium is studied which is similar to coalition-proof Nash equilibrium but considers correlated strategies. All these solution concepts consider only those deviations where all the members of the deviating coalition can benefit from the deviation. This is due to the natural assumption that a player would not deviate with a coalition if he has no incentive in doing so. However, recently \citet{Abraham-2006,Abraham-2008} proposed the solution concept of \textit{k-resilient} Nash equilibrium which also considers the deviations in which even one member of the coalition can benefit. The reasons to consider such deviations as given in \cite{Abraham-2006} have been discussed in the previous section. \citet{Abraham-2006,Abraham-2008} prove that \textit{k-resilient} Nash equilibria exist in rational secret sharing and multi-party computation games with mediators, with some bounds on the values of \textit{k} and \textit{n} (total number of players). They also give various bounds and conditions on when can the mediator be simulated using cheap talk. In all the above works, some kind of prior communication is assumed before the deviation. However, we feel that deviation by multiple players is also possible without any prior communication. This is due to the risk taking behaviour that players may exhibit as explained in the previous section.
 
\section{Organization of the Thesis}

In Chapter 2, we discuss non-deterministic multi-player concurrent reachability games and we prove some properties that characterize \textit{k-resilient} Nash equilibria in these games. We then use these properties to develop the algorithm to check the existence of \textit{k-resilient} Nash equilibrium in finite non-deterministic multi-player concurrent reachability games when only pure strategies are allowed. The properties that we prove also contain all the necessary information to compute a \textit{k-resilient} Nash equilibrium if it exists. Chapter 3 discusses multi-player timed concurrent reachability games and \textit{k-resilient} Nash equilibrium in these games. Chapter 4 gives a brief conclusion of the thesis and discusses some possible future extensions of the work in this thesis. In Appendix A, we give a brief review of the theory of timed automata from \cite{1} required for proper understanding of our work in timed games.