\chapter{Conclusion and Future Work}
In this thesis, we considered non-deterministic multi-player concurrent reachability games in untimed and timed settings and we studied the solution concept of \textit{k-resilient} Nash equilibrium in these games. We proved some properties that characterize \textit{k-resilient} Nash equilibria in these games when only pure strategies are allowed. These properties were then used to develop algorithms for checking existence of \textit{k-resilient} Nash equilibrium in finite non-deterministic multi-player concurrent reachability games and in multi-player timed concurrent reachability games when only pure strategies are allowed. Our algorithms run in time, doubly exponential with respect to the size of input and hence are in 2-EXPTIME. Further, the properties that we proved also contain all the necessary information to compute a \textit{k-resilient} Nash equilibrium if it exists and hence we gave a generic method to compute a \textit{k-resilient} Nash equilibrium if it exists.

In particular, we generalized the concept of \textit{suspect players} \cite{BBM-concur10,BBM-report,BBMU-fsttcs11,Romain-phd} and \textit{repellor sets} \cite{BBM-concur10,BBM-report,BBMU-fsttcs11} to \textit{k-suspect coalitions} and \textit{k-repellor sets} respectively. We also extended the notion of \textit{secure moves} \cite{BBM-concur10,BBM-report} to \textit{k-secure moves}. We defined the notion of transition systems $k-S_{G}(P, X)$ where $P \subseteq Agt$ and $X \subseteq 2^{Agt}_{k}$ (here $Agt$ is the set of players (agents) in the game $G$ and $2^{Agt}_{k}$ is set of all subsets of agents of size at most $k$) which are a generalization of the transition systems $S_{G}(P)$ as defined in \cite{BBM-concur10,BBM-report} but extended in a natural way to suit our setting. We proved various properties concerning \textit{k-suspect coalitions}, \textit{k-repellor sets}, \textit{k-secure moves} and the transition systems $k-S_{G}(P, X)$ that characterize \textit{k-resilient} Nash equilibria in non-deterministic multi-player concurrent reachability games.

The main result that we proved is that there exists a \textit{k-resilient} pseudo-Nash equilibrium in a non-deterministic multi-payer concurrent reachability game $G$ from a state $s$ with a payoff vector $v_{Agt}$, if and only if, letting $P$ be the set of players whose payoffs are $0$ in the payoff vector $v_{Agt}$, there is an infinite path $\pi$ in the transition system $k-S_{G}(P, 2^{Agt}_{k})$ which starts in $s$ and visits $\Omega(A)$ for every $A$ not in $P$ (here $\Omega(A)$ is a subset of states which is the reachability objective of player $A$). This result gave us a necessary and sufficient condition for existence of \textit{k-resilient} pseudo-Nash equilibrium in a non-deterministic multi-player concurrent reachability game and was used to develop the algorithm for checking existence of \textit{k-resilient} pseudo-Nash equilibrium in finite non-deterministic multi-player concurrent reachability games when only pure strategies are allowed. We also gave a generic method to compute a \textit{k-resilient} pseudo-Nash equilibrium if it exists as the various properties that we proved contain all the necessary information for computing a \textit{k-resilient} pseudo-Nash equilibrium if it exists.

In order to decide the existence of \textit{k-resilient} pseudo-Nash equilibrium in multi-player timed concurrent reachability games, we used the classical region based abstraction and constructed the region game. The construction of the region game is same as in \cite{BBM-concur10,BBM-report}. We proved that the translation of a timed game to its associated region game preserves a \textit{k-resilient} pseudo-Nash equilibrium (if any) and hence the existence of \textit{k-resilient} pseudo-Nash equilibrium in a timed game can be decided by checking the existence of \textit{k-resilient} pseudo-Nash equilibrium in the corresponding region game. As a region game is a finite non-deterministic multi-player concurrent reachability game, we can apply the algorithm that we developed for checking existence of \textit{k-resilient} pseudo-Nash equilibrium in these games. Further, the translation proposed (from a timed game to its associated region game) can also be used to compute a \textit{k-resilient} pseudo-Nash equilibrium if it exists.

Our algorithms, constructions and proofs are a natural generalization of those in \cite{BBM-concur10,BBM-report} used to characterize Nash equilibria in non-deterministic multi-player concurrent reachability games. Our algorithms take time, doubly exponential with respect to the size of input and hence are in 2-EXPTIME.

Some possible future extensions of this work are:
\begin{enumerate}
\item We have not been able to prove the optimality of our algorithms. It may be interesting to check if the algorithms can be made better using non-determinism or if 2-EXPTIME (complexity class of proposed algorithms) is actually the lower bound for the problems.
\item We have considered games where only pure strategies are allowed. The work may be extended to games where players can have mixed strategies.
\item We have considered games with reachability objectives only. The work may be extended to study \textit{k-resilient} Nash equilibria in games with other qualitative objectives like safety objectives, B{\"u}chi objectives, co-B{\"u}chi objectives, Rabin objectives and parity objectives. 
\end{enumerate}